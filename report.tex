\documentclass{article}
\title{Diffusion and Score-based Generative Models}
\author{Simone Petruzzi-1811872 Domenico Tersigni-}
\date{August 2023}
\usepackage[margin=1.5in]{geometry} % Adjust the values as per your requirements
\usepackage{amssymb}
\begin{document}
   \maketitle
   \section{Introduction}
	Given a dataset $\{ x_1,x_2,...,x_n \}$ where each point is drawn independently from an underlying distribution $p(x)$, the goal of generative modelling is to fit a model to the data distribution, such that we can synthetize new data points at will by sampling from the model of the distribution.\newline
	Let's say $p_{data}$ is our model and $x_i$ $\sim$ $p_{data}$. We want to find a single probability distribution, minimizing the distance from $p_{data}$ to $p_{\theta}$. Afterwards we can generate samples from $p_{\theta}$. In order to estimate the model we can use as starting point a Gaussian distribution (basically a computational graph composed by two layers). However a single Gaussian is too simple and we want to leverage on a bigger and deeper computational graph. Thus we want to use deep neural networks to represent complex data distributions. This kind of architectures typically convert high dimensional input into one dimensional output $f_{\theta}(x)$, this output should be converted into a probability density and for this reason we take the exponential of the output in order to make it always positive. Then we normalize by means of a constant $Z_{\theta}$ obtaining:\newline
	\begin{equation}
	p_{\theta}(x) = \frac{e^{f_{\theta}(x)}}{Z_{\theta}}
	\end{equation}
	$Z_{\theta}$ by definition should be computed by evaluating the high dimensional integral of the exponential function of out $\theta$, over all possible values of x in the space:
	\begin{equation}
	Z_{\theta}= \int_{}^{} e^{f_{\theta}(x)} \,dx 
	\end{equation}
	In case of deep neural networks the computation of this integral becomes intractable (NP-complete problem). In order to tackle with the intractability of the normalizing constant, this proposal aims to work with \textbf{score functions}.
	\newpage
   \section{Score-based generative modeling}
	The score funcion is a vector field that gives the direction where the density fucntion grows most quickly. We define the score of a probability density $p(x)$ to be:
	\begin{equation}
	p(x) = \nabla_{x}\ log\ p(x)
	\end{equation}
	Score function is good since in anytime given the density function we can compute the score function very easily by simply texting the dervative. Conversely given the score function we can recover the density function in principle by computing the integral. Thus, we argue that score function mantains preseved all mathematical information of the density function, being computationally more easy to deal with. Indeed with score functions we don't have any normalization restriction:
	\begin{equation}
	\nabla_{x}\ log\ p_{\theta}(x) = \nabla_{x}f_{\theta}(x)\ - \nabla_{x}\log Z_{\theta}
	\end{equation}
	the term $\nabla_{x}\log Z_{\theta}$ is always 0 because it is the gradient of a constant. Thus the score function is the gradient of the deep neural network $f_{\theta}(x)$. We define the score model $s_{\theta}(x)$ as:
	\begin{equation}
	\nabla_{x}f_{\theta}(x) = s_{\theta}(x)
	\end{equation}
	and we want to develop a technique that allows us to train $s_{\theta}(x)$, in order to estimate the underlying score function from a limited set of training data points
	\subsection{Score matching}
	We must train our score model to be close to our ground-truth data score function. Mathematically we can capture the distance between the two vector fields of score by the \textbf{Fisher divergence objective}:
	\begin{equation}
	\frac{1}{2}E_{p_{data}(x)}[||\nabla_{x}\ log\ p_{data}(x) - s_{\theta}(x) ||_{2}^{2}]
	\end{equation}		
	However Fisher divergence can not be computed since we don't know the ground trith value of the data score function $\nabla_{x}\ log\ p_{data}(x)$. There is a way for addressing this challenge and the method is called \textbf{score matching}. Score matching uses integration by parts of Gauss' theorem to convert Fisher divergence into the following equivalent objective:
	\begin{equation}
	E_{p_{data}(x)}[\frac{1}{2}||s_{\theta}(x)||_{2}^{2} + trace(\nabla_{x}\ s_{\theta}(x))]
	\end{equation}
	where $\nabla_{x}\ s_{\theta}(x)$ denotes the Jacobian of $s_{\theta}(x)$. We call it score matching objective, and it is equivalent to the fisher divergence up to a costant. Since constants do not affect optimization, their score matching objective defines the same optimum as the Fisher divergence. In score matching there is no dependency on the score function of the data distribution anymore. Moreover the expectation in score matching can be efficiently be approximated by using empirical mean:
	\begin{equation}
	E_{p_{data}(x)}[\frac{1}{2}||s_{\theta}(x)||_{2}^{2} + trace(\nabla_{x}\ s_{\theta}(x))] \approx \frac{1}{N} \sum_{i=1}^{N}[\frac{1}{2}||s_{\theta}(x)||_{2}^{2} + trace(\nabla_{x}\ s_{\theta}(x_i))]
	\end{equation}
	again this is not scalable to compute since we need one forward propagation to compute the first element of the score function output and we need a backpropagation to compute the first element on the diagonal of its Jacobian. This procedure has to be repeated multiple times until we recovered all diagonal elements on the Jacobian (trace is given by the sum over the diagonal elements). Thus, number of backpropagations needed is proportional to the dimenstionality of our data points. It follows that the whole procedure when modelling high dimensional data like images, requires a lot of backpropagations making this naive score matching approach not scalable.
	\subsection{Sliced score matching}
	This approach aims to solve the problem of naive score matching: the basic intuition is that we want to convert high dimensional problem to a one-dimensional one (since it is much easier to solve). \\
	The idea is to leverage on random projection, this is done in order to approximate $trace(\nabla_{x}\ s_{\theta}(x))$ and get one-dimensional scalar fields. Again to capture this intuition we use sliced Fisher divergence:
	\begin{equation}
	\frac{1}{2}E_{p_{v}}E_{p_{data}(x)}[(\mathbf{v}^{T}\nabla_{x}\ log\ p_{data}(x) - \mathbf{v}^{T}s_{\theta}(x) )^{2}]
	\end{equation}
	also there we apply integration by parts and obtain sliced score matching:
	\begin{equation}
	E_{p_{v}}E_{p_{data}(x)}[\mathbf{v}^{T}\nabla_{x} s_{\theta}(x) \mathbf{v}\ +\frac{1}{2}||s_{\theta}(x)||_{2}^{2}]
	\end{equation}
	$\mathbf{v}^{T}$ $\nabla_{x}s_{\theta}(x)\mathbf{v}$ is much more scalable to compute. It is not hard to find that we can rewrite:
	\begin{equation}
	\mathbf{v}^{T}\nabla_{x}s_{\theta}(x)\mathbf{v} = \mathbf{v}^{T}\nabla_{x}(\mathbf{v}^{T}s_{\theta}(x))
	\end{equation}
	we just need one forward propagation to get the output $s_{\theta}(x)$ and then we can directly compute the inner product between $\mathbf{v}^{T}$ and $s_{\theta}$.\\
	Sliced score matching provides score estimation for the original $unperturbed$ data distribution.
	\subsection{Denoising score matching}
	Another approach is denoising score matching which estimates the scores of $perturbed$ data distribution.\\
	Indeed, the idea here is to perturb the original data distribution employing a perturbation kernel $q_{\sigma}$, then estimate the score function of the noisy data density instead of the original one.
	\begin{equation}
	p_{data}(x)\ \longrightarrow q_{\sigma}(\tilde{x}|x)\ \longrightarrow q_{\sigma}(\tilde{x})
	\end{equation}
	We employ score matching to estimate the score of the perturbed data distribution $q_{\sigma}(\tilde{x})$, and when the perturbation is very small, we can approximately consider the score function of the noisy data density as equivalent to the score function of the noise-free data density. The denoising score matching objective function has been proved equivalent to the following
	\begin{equation}
	\frac{1}{2} \mathbb{E}_{p_{data}(x)} \mathbb{E}_{q_{\sigma}(\tilde{x}|x)}[||s_{\theta}(\tilde{x}) - \nabla_{\tilde{x}}logq_{\sigma}(\tilde{x}|x)||_{2}^{2}]
	\end{equation}
	in this approach we must take care of the trade-off between the quantity of the noise injected. If you want to work well in estimating score functions of noise-free data densities, you need a very small $\sigma$. However when $\sigma$ is too small, the variance of the objective function becomes bigger and eventually explodes. 
   \subsection{Langevin dynamics}
   Now that we have trained our score model to estimate the underlying score function $s_{\theta} \approx \nabla_{x} log p(x)$ we want a method for generating new data points from the given vector field of score functions.\\
   Langevin dynamics is an iterative procedure that allows to draw samples from our score model. The procedure is as follows:
   \begin{itemize}
   	\item Initialize the sample from an arbitrary prior distribution: $x^{o} \sim \pi(x)$, then repeat this for $z^{t} \leftarrow 1,2,...,T\ z^{t} \sim \mathcal{N}(0,I)$
	\item In each sampling step, first generate a random Gaussian vector from the standard Gaussian distribution, then modify x according to the following recurrence:
	\begin{equation}
	x_{t} \leftarrow x^{t-1} + \frac{\epsilon}{2} \nabla_{x} log\ p(x^{t-1})+\sqrt{\epsilon}\ z^{t}
	\end{equation}
   \end{itemize}
   We basically update the previous sample using our score function, plus a scaled version of the Gaussian noise vector. If $\epsilon \rightarrow 0$ and $T \rightarrow \infty$, we are guaranteed to have $x^{t} \sim p(x)$.\\
   It becomes very natural to replace the score function in Langevin dynamics with our score model ($s_{\theta}(x) \approx \nabla_{x}log\ p(x)$), then we can generate new data samples, defining a new generative model.
   \newline
   \subsection{Naive score-based generative modeling and its pitfalls}
   This procedure has drawbacks: we have at first problem in estimating correctly the score function in low data density regions. In those regions we usually don't have enough samples to estimate the score function accurately. Remember that score matching minimizes the expected squared error of the score estimates, $i.e., \frac{1}{2}\mathbb{E}_{p_{data}(x)}[||\nabla_{x}\ log\ p_{data}(x) - s_{\theta}(x) ||_{2}^{2}]$. In practice the expectation is always estimated using i.i.d samples $\{ x_{i}\}_{i=0}^{N} \sim p_{data}(x)$ and when there is a lack of them, the estimation is inaccurate.\\
   \newline
    Also, Langevin dynamics,  when two modes of the data distribution are separated by low density regions, will be not able to recover the relative weights of these two modes in reasonable time. Langevin dynamics can produce correct samples in theory, but it may require a very small step size and  a very large number of step to mix, increasing drastically the computation time.
    \newpage
    \section{Score-based generative modeling with multiple noise perturbations}
    In the previous section we talked about score-based generative modeling and it's problems. The main problem of that approach is bad estimation of the score function in low data density regions. We can address this challenge by injecting Gaussian noise to perturb our data data points and then doing the exact same procedure on the perturbed data. After having added enough Gaussian noise we provide useful directional information for Langevin dynamics, also score functions of noisy data densities are much easier to be estimated accurately. \\
    \newline
    However by simply injecting Gaussian noise we don't solve the problem, because after perturbing data points, the noisy data densities distances, are no longer good approximation for the original data density. How to choose the right level of noise ? To solve this, we can use multiple noise levels simultaneously and consider a noise conditional score model. This model takes noise level $\sigma$ as additional input dimension to the model and the output corresponds to the score function of the perturbed data density by $\sigma$.  We perturb the data distribution $p(x)$ with a Gaussian noise $\mathcal{N}(0,\sigma_{i}^{2}I), i=1,2,...,L$, obtaining the noise-perturbed distribution:
   
   
\end{document}